\section{OBM 2020 problema 1}

Esse problema é análogo ao fato que a função $f(n)$ que conta as maneiras de escrever $n=x^2+y^2$ com $x,y$ inteiros, tem limite

$$\frac{f(1)+f(2)+\dots+f(n)}{n}\to\pi$$

quando $n\to\infty$, isso por que o numerador é o número de pontos no plano com ambas as coordenadas inteiras contidos no circulo de raio $\sqrt{n}$ centrado na origem. Essa quantidade se aproxima da área do circulo e portanto o limite é $\pi$. Da mesma maneira, o limite da questão é dado pelo volume do elipsoide $2x^2+3y^2+5z^2=R$, que é

$$\frac{4}{3}\pi\sqrt{2}\sqrt{3}\sqrt{5}R^{3/2}.$$

\section{OBM 2019 problema 5}

Com a expansão em frações parciais:

$$\frac{2}{4n^2+8n+3}=\frac{1}{2n+1}-\frac{1}{2n+3}$$

Fica claro que a soma é telescópica e só o primeiro termo não cancela.

\section{O incrível problema dos reis caminhantes}

